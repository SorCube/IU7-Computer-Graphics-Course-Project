\chapter*{Заключение}
\addcontentsline{toc}{chapter}{Заключение}
\vspace{-0.5cm}\hspace{0.6cm}В ходе выполнения курсового проекта было изучено и реализовано моделирование развевающегося на ветру флага. Была определена и описана предметная область работы, рассмотрены существующие алгоритмы для удаления невидимых линий и поверхностей, из которых был выбран наиболее подходящий алгоритм для решения поставленной задачи. Был спроектирован пользовательский интерфейс и было реализовано требуемое программное обеспечение.

\vspace{0.3cm}В аналитическом разделе был произведен анализ предметной области, алгоритмов, необходимых для решения поставленной задачи. В конструкторском разделе были подробно описаны методы и алгоритмы, было дано описание принципов работы камеры наблюдателя. В технологическом разделе были предъявлены требования к программному обеспечению, написаны средства реализации. Также были перечислены и описаны основные типы и структуры данных, использующиеся в проекте. Приведены листинги структур данных, реализации некоторых функций. Приведен пользовательский интерфейс с подробным описанием отдельных блоков. В экспериментальном разделе было проведено исследование скорости работы программы с двумя разными способами хранения данных.

\newpage

\begin{thebibliography}{3}
	\addcontentsline{toc}{chapter}{Литература}
	\bibitem{web:render}
	Виталий Якин. Рендер изображения [Электронный ресурс] - Режим доступа: https://render.ru/ru/yakin/post/11353, свободный - (17.10.2019)
	
	\bibitem{book:rojers}
	Роджерс Д. Алгоритмические основы машинной графики: Пер. с англ. -- М.:Мир, 1989 - 512 с. ISBN 5-03-000476-9.
	
	\bibitem{machinegraph}
	Белорусский государственный университет. Машинная графика, Computer Graphics [Электронный ресурс] - Режим доступа: https://bsu.by/Cache/Page/353573.pdf, свободный - (20.11.2019)
	
	\bibitem{kurov}
	Куров А.В. Курс лекций по машинной графике. - М., 2019
	
	\bibitem{book:modernopngl}
	Боресков А. В. Программирование компьютерной графики. Современный OpenGL. – М.: ДМК Пресс, 2019. – 372 с. ISBN 978-5-97060-779-4
	
	\bibitem{book:languageshader}
	Вольф Д. OpenGL 4. Язык шейдеров. Книга рецептов / пер. с англ. А. Н. Киселева. – М.: ДМК Пресс, 2015. – 368 с. ISBN 978-5-97060-255-3
	
	\bibitem{web:keyframe}
	Акимов С.В. Введение в анимацию [Электронный ресурс] - Режим доступа: http://structuralist.narod.ru/it/flash/animation.htm, свободный - (Дата обращения: 20.11.2019 г.)
	
	\bibitem{web:cpp}
	Документация C++ [Электронный ресурс]. - Режим доступа:
	https://cppreference.com/, свободный. (Дата обращения: 20.11.2019 г.)
	
	\bibitem{web:qt}
	Документация Qt [Электронный ресурс]. - Режим доступа:
	https://doc.qt.io/, свободный. (Дата обращения: 20.11.2019 г.)
	
	
\end{thebibliography}