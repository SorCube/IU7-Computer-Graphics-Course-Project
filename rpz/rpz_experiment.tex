\chapter{Экспериментальный раздел}
	
\vspace{-0.5cm}\hspace{0.6cm}В данном разделе будет приведено экспериментальное исследование временных затрат разработанного программного обеспечения в зависимости от числа потоков, используемых для отрисовки финального изображения в ходе алгоритма z-буфера. 


\section{Исследование скорости работы алгоритма}
\hspace{0.6cm} Для исследования скоростных характеристик был использован компьютер на базе процессора Intel Core i5 с оперативной памятью 12ГБ. Модуль тестирования запускался под операционной системой Windows 10.

\hspace{0.6cm} В таблице представлены данные, полученные в ходе замера времени работы алгоритма отображения одного и того же объекта с одинаковыми параметрами анимации при разном количестве потоков.

\begin{table}[ht!]
	\begin{center}
		\caption{Время работы алгоритма при разном числе потоков}
		\pgfplotstabletypeset[
		col sep=semicolon,
		string type,
		columns/Threads/.style={column name=Количество потоков, column type={|c}},
		columns/Time/.style={column name=Время в тиках, column type={|c|}},
		every head row/.style={before row=\hline,after row=\hline},
		every last row/.style={after row=\hline},
		]{time.csv}
	\end{center}
\end{table}

\begin{figure}[ht!]
	\begin{center}
		\begin{tikzpicture}
		\begin{axis}
		[%title = График зависимости времени работы алгоритма от числа потоков,
		table/col sep = semicolon,
		xlabel={Число потоков},
		ylabel={Время в тиках},
		ymin = 0,
		legend pos=outer north east,
		ymajorgrids=true,
		grid style=dashed]
		\addplot[color=blue, mark=*] table[x={Threads}, y={Time}] {time.csv};
		\end{axis}
		\end{tikzpicture}
		\caption{График зависимости времени работы алгоритма от числа потоков}
	\end{center}
\end{figure}

\hspace{0.6cm} Очевидно, что программа работает быстрее с большим числом потоков. По графику можно заметить, что время уменьшается практически линейно для небольшого числа потоков.

\section{Вывод}
\hspace{0.6cm} В данном разделе было проведено исследование временных затрат разработанного программного обеспечения в зависимости от количества потоков. Эксперимент показал, что многопоточная версия работает значительно быстрее однопоточной, в частности, работа алгоритма с 4 потоками почти в 2 раза быстрее однопоточной.

