\tableofcontents
	
\chapter*{Введение}
\addcontentsline{toc}{chapter}{Введение}

\vspace{-0.5cm}\hspace{0.6cm}Рендеринг — это процесс создания изображения или последовательности из изображений на основе двухмерных или трёхмерных данных. Данный процесс происходит с использованием компьютерных программ и зачастую сопровождается сложными и комплексными техническими вычислениями, которые ложатся на вычислительные мощности компьютера или на отдельные его комплектующие части.

\vspace{0.3cm}Процесс рендеринга присутствует в разных сферах профессиональной деятельности: киноиндустрия, видеоблогинг, игровая индустрия, телеиндустрия и т. д.. Зачастую, рендер является последним или предпоследним этапом в работе над проектом, после чего работа считается завершенной или же нуждается в небольшой постобработке. Также стоит отметить, что нередко рендером называют не сам процесс рендеринга, а скорее уже завершенный этап данного процесса или его итоговый результат\cite{web:render}.

\vspace{0.3cm}Если мы говорим о задаче рендеринга изображения, то чаще всего имеем в виду рендеринг в трехмерной графике, так как это задача является самой востребованной в этой сфере. Это связано с тем, что как такового трехмерного измерения в компьютерной графике не существует и работа ведется с двухмерным изображением - с экраном.

\vspace{0.3cm}В качестве такого изображения для данной работы была взята анимация развевающегося на ветру флага. Цель работы заключается в моделировании и визуализации данного изображения в трёхмерном пространстве.